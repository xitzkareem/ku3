% Opg 2
\section{Find the greates common divisor using the algorithm we learned, show the steps then 
express d as linear combination of m and n}
To solve this problem we will use the following equation, which was explained in problem 1
\begin{equation}
	m=q*n+r 
\end{equation}
% Opg 2.a
\subsection{ m=38, n=14}
\begin{equation}
	38 = q_0 \cdot 14 +r \to q = 2 , r= 10
\end{equation}
\begin{equation}
	14= q_1 \cdot 10 + r \to q_1 = 1 , r =4
\end{equation}
\begin{equation}
	10 = q_3 \cdot 4 + r \to  q_3 = 2 , r = 2 
\end{equation}
\begin{equation}
	4 = q_4 * 2 +r \to q_4 = 2 , r = 0 
\end{equation}

The gcd is the last remainder that is not 0. To write it in linear combination we have to solve for the following : 
\begin{equation}
	 2 = x  \cdot 14 + y \cdot 38 \dots
\end{equation}
% Opg 2.b
\subsection{m=117, n=69}
 
\begin{equation}
	117 = q_0 * 69 +r
\end{equation}





% Opg 3
\section{Consider the following algorithm}
\begin{minted}{python3}
j := 1 
while (j<=n) {
	A[j] := 0
	for i := j downto 1 {
		A[j] := A[j]+i*i 
	}
	j := j+1
	}
\end{minted}

% Opg 3.a
\subsection{Explain what the algorithm does, and what are the number stored and computed in A}

Let start with the easier part, what are the values stored in array A, from j to n. 
The values stored in A are the sum of squares of j downto 1, meaning that for A[2], A[2 ] is equal to $2^2 + 1^2= 5$. 

j is an index variable showing our position in the array. This algorithm has 2 loops, the outer while-loop runs n times,
going through every element in the array. The inner for-loop, that also run n times,  is used to update the values of the array from j to n. Initially 
A[j:n] = 0, a condition/ command specified at the begining of the outer-loop.  The inner for-loop takes A[j] and iterates throught it i times, until i hits 1.
At each iteration in the for loop, the value of A[j] is updated, then the updated value is used for the next iteration, once i hits 1, the for-loop terminates, and 
j is updated to j+1. 
To reduce abstraction, suppose that n=3 and execute the algorithm at j=3 manually.
  j=3
 \begin{equation}
 A[3] =0 \to j \le n (true, continue)	
 \end{equation} 
 we enter for-loop
  \begin{equation}
	A[3] = 0+3*3=9 \to A[3']= A[3] + 2*2= 13 \to A[3''] = A[3'] + 1*1  = 13+1= 14 
  \end{equation}
  \begin{equation}
 i=1 \textsl{(for-oop terminates)} \to j=3+1 = 4 \to  4 \le 3 \textsl{(j larger than, outer-loop terminates)}
  \end{equation}



% Opg 3.b
\subsection{Provide an asymptotic analysis  of the running time as function of the array size n} 
    
We can analyse the running time of the algorithm by looking at the inner and outer loop. Assuming that every operation takes a constant time
c, we can avoid detailed analysis and look at the recursive operations. The outer-loop will run n-times, and for each j in the while loop run, the for loop runs 
j times. This means that if j = 3, the for loop runs 3 times, j=n, the for-loop with run n-times.The complexity of this such a relationship is n*n = n² .
Asymptotic analysis refers to the behavior of the algorithm as n grows to infinity. 

\begin{equation}
O(n^2 )
\end{equation}





% Opg 3.c
\subsection{Improve the code to run faster while retaining the same functionality, analyse the time complexity }
This algorithm should run faster considering that it only has one loop that runs n times, rather than  2-loops running n times each.

\begin{minted}{python}
A[1:n]
for i = 2 up to n :
	if i <= n :
		A[i] = i*i +A[i-1]
	i = i+1 	
\end{minted}


